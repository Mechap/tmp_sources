\section*{Type deduction}

C++98 had a single set of rules for type deduction: the ones for function templates. C++11 modified that ruleset a bit and adds two more, one for \mintinline{c++}|decltype| and one for \mintinline{c++}|auto|. C++14 then extends the usage contexts in which \mintinline{c++}|auto| and \mintinline{c++}|decltype| may be employe| may be employed. \\

One can think of a template function as looking like this :

\begin{minted}{c++}
template <typename T>
void f(ParamType param);

f(expr);
\end{minted}

During compilation, compilers use \textit{expr} to deduce two types: one for \textit{T}, and one for \textit{ParamType}. These types are frequently different, because \textit{ParamType} often contains adornments, e.g, const or reference qualifers. \\

It's natural to expect the type deduced for \textit{T} is the same as the type of the argument passed to the function, i.e., that \textit{T} is the type of \textit{expr}. The type deduced for \textit{T} is dependent not only on the type of \textit{expr}, but also on the form of \textit{ParamType}. \\

They are three cases :

\begin{itemize}
    \item[(i)] \textit{ParamType} is a pointer or reference type, but not a forwarding reference (forwarding refernces will be explained in a dedicated section).

    \item[(ii)] \textit{ParamType} is a forwarding reference.
    \item[(iii)] \textit{ParamType} is neither a pointer nor a reference.
\end{itemize}

%\textcolor{black}{\rule{16cm}{0.2mm}}
\hrulefill

\subsubsection*{Case 1: \textit{ParamType} is a reference or pointer, but not a forwarding reference}

The easiest situation is when \textit{ParamType} is a reference or pointer, but not a forwarding reference.

\begin{minted}{c++}
template <typename T>
void f(T &param);

int x = 27;        // decltype(x) == int
const int cx = x;  // decltype(cx) == const int
const int &rx = x; // decltype(rx) == const int&, i.e., a reference to x as a const int

f(x);  // 1
f(cx); // 2
f(rx); // 3
\end{minted}

In the first call, \textit{T} is deduced as \mintinline{c++}|int|, and \textit{param}'s type as \mintinline{c++}|int&|. In the second and third calls, notice that because \textit{rx} and \textit{cx} designate const values, \textit{T} is deduced to be const int, thus yielding a paramater type of \mintinline{c++}|const int&|. In the third example, note that even though \textit{rx}'s type  is a reference, \textit{T} is deduced to be a non-reference. That's because \textit{rx}'s reference-ness is ignored during type deduction. \\

If we change the type of \textit{f}'s parameter from \mintinline{c++}|T&| to \mintinline{c++}|const T&|, things change a little bit. The constness of \textit{cx} and \textit{rx} is still respected but because we're now assuming that \textit{param} is a reference-to-const, there's no need for \mintinline{c++}|const| to be deduced as part of \textit{T}.

\subsubsection*{Case 2: \textit{ParamType} is a forwaring reference}

Things are less obvious for templates taking forwaring reference parameters\footnote{which are also called "universal references"}. Such parameters are declared like rvalue references, but they behave differently when lvalue arguments are passed in.

\begin{minted}{C++}
template <typename T>
void f(T &&param);

f(x);  // x is lvalue, so T is int& and param's type is also int&
f(cx); // x is lvalue, so T is const int& and param's type is also const int&
f(rx); // x is lvalue, so T is const int& and param's type is also const int&
f(27); // x is rvalue, so T is int and param's type is int&&
\end{minted}

\begin{itemize}
	\item If \textit{expr}  is an lvalue, both \textit{T} and \textit{param's type} are lvalue references.

	\item If \textit{expr} is an rvalue, \textbf{Case 1} is applied.
\end{itemize}

This logic can be explained with one restriction introduced with references :
\begin{displayquote}
	\textit{A reference can not point to an other reference.}
\end{displayquote}

However, it is sometimes possible to generate similar references :
\begin{minted}{c++}
using int_ref = int&;

int_ref&& a; // decltype(a) -> int_ref& && ?
\end{minted}

In such situations, the specification applies the \textit{reference collapsing rule} :
\begin{itemize}
    \item \mintinline{C++}|&| + \mintinline{C++}|&| = \mintinline{C++}|&|
    \item \mintinline{C++}|&&| + \mintinline{C++}|&| = \mintinline{C++}|&|
    \item \mintinline{C++}|&&| + \mintinline{C++}|&&| = \mintinline{C++}|&&|
\end{itemize}

Thus :
\begin{minted}{c++}
static_assert(std::is_same_v<decltype(a), int&>);
\end{minted}

You can already see that we are in a similar situation with forwarding references.
\begin{minted}{c++}
f(a) // compiler instantiates
     // f<int&>(int && &) which translates to
     // f<int&>(int &) thanks to reference collapsing

f(static_cast<int&&>(a)); // compiler instantiates
                          // f<int&&>(int && &&) which translates to
                          // f<int&&>(int &&) 
\end{minted}

This behaviour is refered in the specification as the \textit{reference collapsing}
where the compiler is required to follow the above rules.
Please note, that such behaviour can appear within syntactical macros.

\begin{minted}{c++}
#define lvalue int&
#define rvalue int&&

lvalue &a;      // decltype(a) → int& + & → int&
lvalue &&b;     // decltype(b) → int& + && → int&

rvalue &c;      // decltype(c) → int && + & → int&
rvalue &d;      // decltype(d) → int && + && → int&&
\end{minted}

\subsubsection*{Case 3: \textit{ParamType} is neither a pointer nor a reference}

When \mintinline{C++}|ParamType| is neither a pointer nor a reference, we are dealing with pass-by-value.

\begin{minted}{c++}
template <typename T>
void f(T param); // param is now passed by value
\end{minted}

That means, that \mintinline{c++}|param| will be a copy of whatever is passed in, i.e., a completely new object.
The fact that \mintinline{c++}|param| will be a new object motivates the rules that govern how T is reduced from \mintinline{c++}|expr|

\begin{itemize}
    \item[(i)] As before, if \mintinline{c++}|expr|'s type is a reference, ignore the reference part.
    \item[(ii)] If, after ignoring \mintinline{c++}|expr|'s reference-ness, \mintinline{c++}|expr| is \mintinline{c++}|const| or \mintinline{c++}|volatile|, ignore that too.
\end{itemize}

Therefore :
\begin{minted}{c++}
int x = 27;         // as before
const int cx = x;   // as before
const int &rx = x;  // as before

f(x);               // T's and param's types are both int
f(cx);              // T's and param's types are again both int
f(rx);              // T's and param's types are still both int
\end{minted}

Please note that even though \mintinline{c++}|cx| and \mintinline{c++}|rx| represent \mintinline{c++}|const| values, \mintinline{c++}|param|'s is not \mintinline{c++}|const|, since it is a completely independent object.
