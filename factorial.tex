\section*{A factorial function}

Now that we have a common understanding of this system, we can start to write programs. A typical implementation of a runtime factorial function can generally be written as:

\begin{minted}{c++}
int factorial(int x) {
    if (x == 0) {
	return 1;
    }

    return x * factorial(x - 1);
}
\end{minted}

Where the factorial definition is :
\begin{center}
	$\displaystyle n! = \prod_{i=1}^{n}i$
\end{center}

Hence, when we pass an integer input value to our factorial function it will return
\mintinline{c++}|value - factorial(value - 1)| until we recursively reach \mintinline{c++}|factorial(0)|.

%\textcolor{black}{\rule{16cm}{0.2mm}} \\
\hrulefill

\begin{paracol}{2}
Since it is a runtime function, the compiler is required to generate machine code:
\begin{minted}{asm}
factorial(int):
    push    rbp
    mov     rbp, rsp
    sub     rsp, 16
    mov     DWORD PTR [rbp-4], edi
    cmp     DWORD PTR [rbp-4], 0
    jne     .L2
    mov     eax, 1
    jmp     .L3

.L2:
	mov     eax, DWORD PTR [rbp-4]
	sub     eax, 1
	mov     edi, eax
	call    factorial(int)
	imul    eax, DWORD PTR [rbp-4]

.L3:
	leave
	ret
\end{minted}

\switchcolumn
Thus for each call to this factorial function, the processor, i.e., the execution unit of your computer will run this set of instructions, even when the input integer value is known at compile time. 

As you may have already guessed we could optimize our factorial function in order to be pre-calculated at compile time when the given input value is also known at compile time.
\end{paracol}

