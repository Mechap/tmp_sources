\section*{Introduction}

Template metaprogramming is a family of techniques to create new types and compute values at 
compile time. Historically, it can be considered as an accident ; it was discovered during the 
process of standardizing the C++ language that its template system appears to be Turing-Complete
, i.e., capable in principle of computing anything that is computable.

The first concrete demonstration of this was a program written by Erwin Unruh, which computed 
prime numbers although it did not actually finish compiling the list of prime numbers was 
part of an error message generated by the compiler on attempting to compile the code.

Template metaprogramming has since advanced considerably and it is now a practical tool for 
library builders in C++, though its complexities mean it is generally not appropriate for the 
majority of applications  or systems programming concepts. \\

In this article, we will dive into this subject and learn about the new metaprogramming 
tricks that have been discovered in recent years.

\vspace{60pt}

%\vfill

\textcolor{black}{\rule{16cm}{0.2mm}}

\section*{Assumptions}

This article will assume the reader has a basic arithmetic knowledge and knows the fundamental concepts in programming, although we will be reviewing them in the C++ model. Metaprogramming in C++ is a tough topic both on the abstraction involved and on the complexities introduced by the template syntax. Thus, it is essential to have a correct understanding of how a C++ program works at its fundamental level. \\

